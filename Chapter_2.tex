\sectionbreak \section*{
  \cyrillicfont 
  \fontsize{14pt}{0pt}\selectfont
  \englishfont 
  \redline
  2. ПЛАН ИНДИВИДУАЛЬНОГО ЗАДАНИЯ
}

\titlespace

\subsection*{ 
  \gostTitleFont
  \redline
  2.1 Анализ существующих решений. Описание достоинств и недостатков 
} 

\subtitlespace

{\gostFont

  \par \redline Веб-сервис для распознавания медицинских патологий на основе МРТ-снимков представляет собой сложную систему, объединяющую медицинскую диагностику, искусственный интеллект и веб-технологии. Основная цель проекта — предоставить врачам и пациентам инструмент для автоматизированного анализа медицинских изображений, что позволит ускорить процесс диагностики и повысить её точность. В рамках данной главы проводится анализ существующих решений в области автоматизированной диагностики медицинских изображений, рассматриваются их достоинства и недостатки, а также выявляются ключевые проблемы, которые необходимо решить в рамках данного проекта.

  \par \redline В настоящее время существует множество приложений и систем, связанных с автоматизированной диагностикой медицинских изображений, таких как МРТ, КТ и рентген. Наиболее популярные из них включают Google DeepMind Health, IBM Watson Health, Aidoc, Zebra Medical Vision, ITK-SNAP, 3D Slicer, U-Net, CheXpert и MONAI. Эти решения предлагают различные подходы к анализу медицинских изображений, начиная от классических алгоритмов обработки изображений и заканчивая современными методами машинного и глубокого обучения.

  \par \redline Google DeepMind Health – это платформа, которая использует глубокое обучение для анализа медицинских изображений, включая диагностику заболеваний глаз и рака. Основное преимущество DeepMind Health – высокая точность диагностики, сравнимая с экспертами-врачами. Однако система требует больших объёмов данных для обучения и значительных вычислительных ресурсов. Кроме того, интерпретируемость модели остаётся проблемой, так как врачи не всегда могут понять, на основе каких признаков модель делает выводы.

  \par \redline IBM Watson Health – это платформа, которая применяет машинное обучение для анализа медицинских данных, включая изображения. Основное достоинство платформы – её универсальность: она может работать с различными типами данных, включая медицинские изображения, текстовые отчёты и генетическую информацию. Однако, как и в случае с DeepMind, Watson Health требует больших объёмов данных для обучения, а её использование может быть дорогостоящим для небольших клиник.

  \par \redline Aidoc – это коммерческая платформа для анализа медицинских изображений на основе искусственного интеллекта. Aidoc интегрируется с системами PACS и помогает врачам в диагностике, автоматически выделяя патологии на снимках. Основное преимущество Aidoc – её готовность к использованию: платформа не требует глубоких знаний в области машинного обучения и легко интегрируется с существующими медицинскими системами. Однако Aidoc ограничена функционалом, предоставляемым разработчиками, и пользователи не могут легко модифицировать алгоритмы.

  \par \redline Zebra Medical Vision – это компания, предлагающая решения для автоматизированного анализа медицинских изображений, включая диагностику рака, сердечно-сосудистых заболеваний и других патологий. Zebra Medical Vision использует глубокое обучение для анализа снимков и предоставляет врачам подробные отчёты. Основное достоинство платформы – её высокая точность и возможность адаптации для различных типов патологий. Однако, как и другие коммерческие решения, Zebra Medical Vision может быть дорогостоящим для небольших клиник, а её использование требует интеграции с существующими системами.

  \par \redline ITK-SNAP – это программа для сегментации медицинских изображений, которая использует классические алгоритмы для выделения областей интереса. ITK-SNAP позволяет врачам вручную настраивать параметры сегментации, что делает её полезной для исследовательских задач. Основное преимущество ITK-SNAP – её интерпретируемость: врачи могут чётко понять, какие именно признаки были использованы для диагностики. Однако программа ограничена в точности и не может автоматически улучшаться с увеличением объёма данных.

  \par \redline 3D Slicer – это инструмент для анализа медицинских изображений, который включает в себя множество классических методов обработки. 3D Slicer позволяет врачам визуализировать и анализировать трёхмерные медицинские изображения, что делает её полезной для планирования хирургических операций. Основное достоинство 3D Slicer – её универсальность: программа поддерживает различные форматы изображений и может быть адаптирована для различных задач. Однако, как и ITK-SNAP, 3D Slicer требует ручной настройки и не может автоматически обучаться на данных.

  \par \redline U-Net – это архитектура свёрточной нейронной сети, разработанная для биомедицинской сегментации изображений. U-Net широко используется для анализа МРТ и КТ, показывая высокую точность в задачах сегментации. Основное преимущество U-Net – её универсальность: одна и та же архитектура может быть адаптирована для различных типов патологий и модальностей изображений. Однако U-Net требует больших объёмов данных для обучения и значительных вычислительных ресурсов.

  \par \redline CheXpert – это модель на основе глубокого обучения для анализа рентгеновских снимков грудной клетки. CheXpert позволяет врачам быстро и точно диагностировать заболевания лёгких, такие как пневмония и рак. Основное достоинство CheXpert – её высокая точность и возможность автоматического обучения на данных. Однако модель требует больших объёмов размеченных данных и может быть чувствительна к шуму и артефактам на снимках.

  \par \redline MONAI – это платформа для разработки и внедрения моделей глубокого обучения в медицинской визуализации. MONAI предоставляет инструменты для создания и обучения моделей, что делает её полезной для исследователей и разработчиков. Основное преимущество MONAI – её гибкость: платформа поддерживает различные архитектуры нейронных сетей и может быть адаптирована для различных задач. Однако MONAI требует глубоких знаний в области машинного обучения и не является готовым решением для клинического использования.

  \par \redline Каждое из этих приложений и систем имеет свои достоинства и недостатки. Основные различия заключаются в подходе к анализу данных (классические методы, машинное обучение, глубокое обучение), функциональности (готовые решения vs. исследовательские инструменты) и требованиях к данным и вычислительным ресурсам. Однако все они направлены на одну цель – улучшение процесса диагностики и повышение точности анализа медицинских изображений.

  \par \redline В отличие от существующих решений, предлагаемый веб-сервис может агрегировать в себе несколько моделей машинного обучения для распознавания медицинских патологий пациента, выбирая наиболее оптимальные подходы в зависимости от типа данных и поставленной задачи. Это позволяет повысить точность диагностики, адаптироваться к различным типам патологий и минимизировать ограничения, связанные с использованием  ИИ-моделей.
  
  \par
}

\subtitlespace

\subsection*{ 
  \gostTitleFont
  \redline
  2.2 Постановка задачи и описание функций разрабатываемой системы 
} 

\subtitlespace

{\gostFont

  \par \redline Разработка веб-сервиса для распознавания медицинских патологий на основе МРТ-снимков представляет собой сложную задачу, которая требует интеграции современных технологий машинного обучения, веб-разработки и медицинской диагностики. Основная цель проекта – создание системы, которая позволит автоматизировать процесс анализа медицинских изображений, предоставит врачам и пациентам удобный инструмент для диагностики и повысит точность выявления патологий. В данном разделе описываются основные задачи, функции и требования к разрабатываемой системе.

  \par \redline Основная задача разрабатываемого веб-сервиса – автоматизированное распознавание медицинских патологий на основе анализа МРТ-снимков. Система должна предоставлять врачам и пациентам возможность загружать медицинские изображения, получать результаты анализа в виде распознанных патологий и рекомендаций по дальнейшим действиям. Для достижения этой цели необходимо решить следующие задачи:

  \par \redline \textbf{1. Разработка модели машинного обучения:}
  \par \redline \hspace{0.3cm} • Создание и обучение модели на основе глубокого обучения для анализа МРТ-снимков;
  \par \redline \hspace{0.3cm} • Обеспечение высокой точности распознавания патологий, сравнимой с врачами;
  \par \redline \hspace{0.3cm} • Поддержка различных типов патологий и модальностей изображений.

  \par \redline \textbf{2. Разработка серверной части:}
  \par \redline \hspace{0.3cm} • Создание API для обработки запросов от клиентской части;
  \par \redline \hspace{0.3cm} • Интеграция с базой данных для хранения данных пользователей, медицинских изображений и результатов анализа;
  \par \redline \hspace{0.3cm} • Обеспечение безопасности и защиты данных пользователей.

  \par \redline \textbf{3. Разработка клиентской части:}
  \par \redline \hspace{0.3cm} • Создание интуитивно понятного интерфейса для загрузки изображений и просмотра результатов анализа;
  \par \redline \hspace{0.3cm} • Реализация интерактивных элементов, таких как формы загрузки, отображение результатов и рекомендаций;
  \par \redline \hspace{0.3cm} • Поддержка различных устройств и браузеров.

  \par \redline \textbf{4. Обеспечение безопасности и конфиденциальности:}
  \par \redline \hspace{0.3cm} • Реализация механизмов аутентификации и авторизации пользователей;
  \par \redline \hspace{0.3cm} • Шифрование данных при передаче и хранении;
  \par \redline \hspace{0.3cm} • Соблюдение нормативных требований, таких как GDPR и HIPAA.

  \par \redline \textbf{Описание функций разрабатываемой системы:}
  \par \redline Разрабатываемый веб-сервис будет предоставлять следующие функции:

  \par \redline \textbf{1. Загрузка и обработка медицинских изображений:}
  \par \redline \hspace{0.3cm} • Пользователи (врачи или пациенты) смогут загружать МРТ-снимки в систему через веб-интерфейс;
  \par \redline \hspace{0.3cm} • Система будет автоматически обрабатывать загруженные изображения, преобразовывая их в формат, подходящий для анализа.

  \par \redline \textbf{2. Анализ изображений с использованием моделей машинного обучения:}
  \par \redline \hspace{0.3cm} • Загруженные изображения будут передаваться в модель машинного обучения, которая проведёт анализ и выявит возможные патологии;
  \par \redline \hspace{0.3cm} • Результаты анализа будут включать в себя описание выявленных патологий, их локализацию и степень тяжести.

  \par \redline \textbf{3. Отображение результатов анализа:}
  \par \redline \hspace{0.3cm} • Результаты анализа будут отображаться в виде текстового описания и визуализации (например, выделение областей с патологиями на изображении);
  \par \redline \hspace{0.3cm} • Пользователи смогут просматривать результаты в удобном формате и сохранять их для дальнейшего использования.

  \par \redline \textbf{4. Генерация рекомендаций:}
  \par \redline \hspace{0.3cm} • На основе результатов анализа система будет предоставлять рекомендации по дальнейшим действиям (например, необходимость дополнительных исследований или консультации с узким специалистом).

  \par \redline \textbf{5. Управление пользователями и данными:}
  \par \redline \hspace{0.3cm} • Система будет поддерживать регистрацию и аутентификацию пользователей (врачей и пациентов);
  \par \redline \hspace{0.3cm} • Пользователи смогут просматривать историю своих запросов и результатов анализа;
  \par \redline \hspace{0.3cm} • Данные пользователей и медицинские изображения будут храниться в защищённой базе данных.

  \par \redline \textbf{6. Масштабируемость и производительность:}
  \par \redline \hspace{0.3cm} • Система будет разработана с учётом возможности масштабирования для обработки большого количества запросов;
  \par \redline \hspace{0.3cm} • Оптимизация производительности для обеспечения быстрой обработки изображений и выдачи результатов.

  \par \redline \textbf{7. Безопасность и конфиденциальность:}
  \par \redline \hspace{0.3cm} • Реализация механизмов шифрования данных при передаче и хранении;
  \par \redline \hspace{0.3cm} • Соблюдение нормативных требований по защите персональных данных и медицинской информации.

  \par \redline \textbf{Требования к системе:}

  \par \redline \textbf{1. Функциональные требования:}
  \par \redline \hspace{0.3cm} • Возможность загрузки и анализа МРТ-снимков;
  \par \redline \hspace{0.3cm} • Поддержка различных форматов медицинских изображений;
  \par \redline \hspace{0.3cm} • Генерация текстовых и визуальных результатов анализа;
  \par \redline \hspace{0.3cm} • Предоставление рекомендаций на основе результатов анализа;
  \par \redline \hspace{0.3cm} • Управление пользователями и данными.

  \par \redline \textbf{2. Нефункциональные требования:}
  \par \redline \hspace{0.3cm} • Высокая производительность и скорость обработки запросов;
  \par \redline \hspace{0.3cm} • Масштабируемость для поддержки большого количества пользователей;
  \par \redline \hspace{0.3cm} • Обеспечение безопасности и конфиденциальности данных;
  \par \redline \hspace{0.3cm} • Совместимость с различными устройствами и браузерами.

  % \par \redline \textbf{3. Технические требования:}
  % \par \redline \hspace{0.3cm} • Использование современных технологий, таких как Java, Python, TensorFlow, Spring Boot, MySQL, HTML, CSS, JavaScript и FreeMarker.
  % \par \redline \hspace{0.3cm} • Поддержка стандартов передачи и хранения медицинских данных (DICOM, NIfTI).
  % % \par \redline \hspace{0.3cm} • Интеграция с медицинскими системами (PACS, EHR).

  % \par \redline \textbf{Заключение:}
  % \par \redline Разрабатываемый веб-сервис для распознавания медицинских патологий на основе МРТ-снимков представляет собой комплексное решение, которое объединяет современные технологии машинного обучения, веб-разработки и медицинской диагностики. Основная задача системы – автоматизация процесса анализа медицинских изображений, что позволит врачам и пациентам получать точные и своевременные результаты диагностики. Система будет предоставлять широкий набор функций, включая загрузку и обработку изображений, анализ с использованием моделей машинного обучения, отображение результатов и генерацию рекомендаций. Учитывая требования к производительности, безопасности и масштабируемости, разрабатываемый веб-сервис станет эффективным инструментом для улучшения качества медицинской диагностики.
  \par
}

\subtitlespace

\subsection*{ 
  \gostTitleFont
  \redline
  2.3 Выбор средств реализации 
} 

\subtitlespace

{\gostFont

  \par \redline Для разработки веб-сервиса распознавания медицинских патологий на основе МРТ-снимков необходимо выбрать подходящие технологии и инструменты, которые обеспечат высокую производительность, масштабируемость, безопасность и удобство использования. В данном разделе обосновывается выбор средств реализации, включая языки программирования, фреймворки, библиотеки, базы данных и другие инструменты.
  
  \par \redline В качестве основных языков программирования выбраны Java и Python для разработки серверной части и алгоритмов машинного обучения соответственно. Java обладает высокой производительностью и масштабируемостью, что важно для обработки большого количества запросов от пользователей. Богатая экосистема фреймворков и библиотек, таких как Spring Boot, упрощает разработку веб-приложений. Java-приложения кроссплатформенны, что позволяет им работать на различных операционных системах, а их надёжность и безопасность критичны для медицинских приложений. Java будет использоваться для разработки серверной части веб-сервиса, включая обработку HTTP-запросов, управление пользователями и интеграцию с базой данных. Python, в свою очередь, отличается простотой и читаемостью кода, что ускоряет разработку и тестирование. Широкая поддержка библиотек для машинного обучения и анализа данных, таких как TensorFlow, Keras и Scikit-learn, а также большое сообщество разработчиков и обширная документация делают Python идеальным выбором для разработки и обучения моделей машинного обучения, а также для обработки медицинских изображений.

  \par \redline В качестве основных фреймворков для разработки серверной части и моделей машинного обучения выбраны Spring Boot и TensorFlow/Keras. Spring Boot упрощает создание веб-приложений на Java, предоставляя готовые решения для работы с базами данных, аутентификации, безопасности и других задач. Поддержка RESTful API позволяет легко интегрировать сервис с другими системами, а высокая производительность и масштабируемость делают Spring Boot подходящим для разработки серверной части веб-сервиса, включая API для взаимодействия с клиентской частью и базой данных. TensorFlow – одна из самых популярных библиотек для машинного обучения, поддерживающая глубокое обучение и работу с большими объёмами данных. Keras, как высокоуровневый API для TensorFlow, упрощает создание и обучение нейронных сетей, а поддержка GPU ускоряет обучение моделей. TensorFlow и Keras будут использоваться для разработки и обучения моделей распознавания патологий на МРТ-снимках.

  \par \redline Для хранения данных пользователей, медицинских изображений и результатов анализа выбрана система управления базами данных MySQL. MySQL обладает надёжностью и стабильностью, являясь одной из самых популярных СУБД с многолетней историей. Поддержка транзакций и ACID-свойств обеспечивает целостность данных, а хорошая интеграция с Java и Spring Boot упрощает разработку. MySQL поддерживает большие объёмы данных и может быть настроена для работы в кластере, что делает её подходящей для хранения данных пользователей, информации о медицинских изображениях и результатах анализа.

  \par \redline Для разработки клиентской части веб-сервиса выбраны HTML, CSS, JavaScript и FreeMarker. HTML и CSS – стандартные технологии для создания веб-страниц, которые поддерживаются всеми современными браузерами. Их простота в использовании и широкие возможности позволяют создавать интуитивно понятный интерфейс. JavaScript позволяет создавать интерактивные элементы на веб-страницах, такие как формы загрузки изображений и отображение результатов анализа. Широкая поддержка библиотек и фреймворков, таких как React или Vue.js, делает JavaScript подходящим для реализации динамических элементов на стороне клиента. FreeMarker – шаблонизатор для Java, который позволяет динамически генерировать HTML-страницы на сервере. Его простота интеграции с Spring Boot и поддержка сложной логики отображения данных делают FreeMarker идеальным для генерации HTML-страниц на основе данных, полученных от сервера.

  \par \redline В качестве дополнительного инструмента для упрощения разработки выбран Git. Git – система контроля версий, которая позволяет отслеживать изменения в коде и сотрудничать с другими разработчиками. Поддержка ветвления и слияния упрощает разработку новых функций, а управление исходным кодом проекта становится более эффективным.

  \par \redline Выбор средств реализации был сделан с учётом требований к производительности, масштабируемости, безопасности и удобству разработки. Java и Spring Boot обеспечивают надёжную основу для серверной части, Python и TensorFlow/Keras – мощные инструменты для машинного обучения, а MySQL – стабильную и масштабируемую базу данных. Клиентская часть, реализованная на HTML, CSS, JavaScript и FreeMarker, обеспечивает интуитивно понятный интерфейс для пользователей. В совокупности эти технологии позволяют создать современный и эффективный веб-сервис для распознавания медицинских патологий.

  \par
}